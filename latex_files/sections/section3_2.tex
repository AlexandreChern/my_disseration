
AmgX is a proprietary software library developed by NVIDIA to accelerate the solution of large-scale linear systems arising from finite element and finite volume discretizations typically found in computational fluid dynamics (CFD) and computational mechanics simulations on NVIDIA GPUs \citep{doi:10.1137/140980260}.
AmgX stands for Algebraic Multigrid Accelerated.
It provides wrappers to work with other libraries like PETSc and programming languages like Julia.

Key features of AMGX include:
\begin{itemize}
    \item Preconditioning: AmgX offers a variety of advanced preconditioning techniques, including algebraic multigrid (AMG), smoothed aggregation and hybrid methods to accelerate the convergence of iterative solvers for sparse linear systems.
    These preconditioners are designed for and tested in real-world engineering problems in collaboration with companies like ANSYS, a provider of leading CFD software Fluent.
    \item Parallelism: AmgX is optimized for NVIDIA GPUs and provides support for OpenMP to allow acceleration via heterogeneous computing and MPI to run large simulations across multiple GPUs and clusters.
    \item Flexibility and Customization: AmgX offers a flexible and extensible framework for configuring and customizing the solver algorithms via JSON files. 
\end{itemize}
The limitation of AmgX is due to its link with NVIDIA. It can not run on GPUs from other vendors, such as AMD and Intel. Some of the latest exascale supercomputers are built with CPUs and GPUs from AMD and Intel.