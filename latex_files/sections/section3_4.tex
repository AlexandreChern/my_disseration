
\section{Fortran}
There are many languages designed for high performance computing.
Traditionally, Fortran has been used to write high performance numerical code.
It is short for "Formula Translation". As the name suggest, it is one of the oldest and most enduring programming languages in scientific computing.
Developed in the 1950s by IBM, it was designed to facilitate numerical and scientific computations, particularly for high-performance computing on mainframe computers.

Fortran was specifically designed for efficient numerical and scientific computing, with optimized operations handling mathematical operations, arrays, and complex computations.
It provides a rich set of built-in functions and libraries for numerical analysis, linear algebra, differential equations, and other mathematical tasks.
It is a statically typed language, meaning that variable types are declared explicitly at compile time and do not change during runtime. This allows compilers to perform extensive type checking and optimization to generate efficient code for execution.

Fortran codes are also highly portable across different computing platforms. While early versions of Fortran (66, 77) were designed for specific hardware architectures, modern Fortran standards, such as Fortran 90, 95, 2003, 2008, and 2018 (formerly 2015) have introduced features that enhance portability and interoperability with other languages and systems. Fortran is also known for its excellent backward compatibility, with newer language standards preserving compatibility with older databases.
This allows legacy Fortran programs to continue running without modification on modern compilers and systems, ensuring long-term viability and support for existing applications, which is very important in scientific research where many simulation codes are built on top of decades of previous work.

Because of these reasons, despite its age, Fortran remains widely used in scientific and engineering computing.

\section{C++}