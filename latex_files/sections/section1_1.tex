% \newline
~\\
% Test: including section 1.1 into the file
% \newline
% \begin{equation}
% y = a*x + b
% \nabla 
% \end{equation}

% This is a sample citation: \cite{schwartz_overview_2012}.
\section{Introduction to earthquakes}
Every year, around ~20,000 earthquakes happen around the world. Some are not noticeable, while others cause huge damage to property and life.
Earthquakes have been recorded for thousands of years, and they are considered as signs or punishments to humans from supernatural powers in many civilizations.
Over the past centuries, with advancements in mathematics, physics, geology, and other natural sciences, we have a more structured understanding of earthquakes today.

An earthquake represents a complex process of fault slip and energy release within the Earth's crust, driven by tectonic forces and resulting in the shaking of the ground and potentially causing damage to structures and infrastructure. 
An earthquake occurs due to the sudden release of accumulated stress along a fault line, resulting in rapid movement known as fault slip. This movement can be described in terms of several key components:

\subsection{Fault rupture}
The earthquake begins with the rupture of the fault, where the stress accumulated along the fault plane exceeds the strength of the rocks, causing them to fracture and slide past each other. This rupture initiates the seismic event.

\subsection{Sesmic waves} 
As the fault ruptures, it generates seismic waves that propagate through the Earth's crust and propagate outward from the fault.
These weaves transmit energy in the form of vibrations, which cause the ground to shake.

\subsection{Slip motion}
Slip motions are the noticeable components of an earthquake movement.
The fault slip during an earthquake involves two types of movements
\begin{itemize}
    \item Primary slip (seismic slip): This is the sudden and rapid movement along the fault plane during the initial rupture. It is usually associated with the intense shaking and damage from the earthquakes
    \item Afterslip: Following the primary slip, there may be additional movement along the fault. This ongoing slip can continue for days, weeks, or even months after the initial earthquake.
\end{itemize}

\subsection{Displacement}
The amount of fault slip during an earthquake is measured in terms of displacements. 
This is the distance that one side of the fault moves relative to the other.
Displacement can be horizontal, vertical, or both.


\section{Seismic and seismic slip}
The slip motion of an earthquake can further be classified into seismic and seismic slips.
\subsection{Seismic slip}
Seismic slip refers to the sudden release of accumulated tectonic stress along a fault plane, resulting in what is often called an earthquake.
This type of fault slip is characterized by rapid and dynamic movement, which generates seismic waves propagating through Earth's crust, causing ground shake and potential damages to infrastructures.
Seismic fault slip occurs when stress accumulated along a fault exceeds the frictional resistance holding the fault surface, causing sudden slip and rupture.
\subsection{Aseismic slip}
Aseismic slip, also known as creep or slow slip, refers to gradual continuous movement along a fault plane without generating significant earthquakes and seismic waves.
Unlike seismic slip, seismic slip occurs at rates that are usually much slower and may not produce noticeable ground shaking or seismic activity.
Instead, seismic slip represents a steady release of tectonic stress along the fault, often occurring in between larger seismic events.
However, the stress could still increase during seismic slip, leading to seismic slip in the future.
Aseismic slip can contribute to the overall deformation of the Earth's crust and could play a potential role in seismic hazard assessments and forecasting.
Modeling the behavior of aseismic slip has been essential in order to understand the nucleation of seismic slip and the mechanism behind multiple cycles of earthquakes.

\subsection{Budget}
In the context of seismology, budget refers to the distribution and allocation of accumulated tectonic stress or energy between seismic and aseismic slip events.
Understanding the balance between seismic and aseismic slip budgets is important for assessing seismic hazard and fault behavior.
It helps researchers and geoscientists to understand the mechanisms governing fault movement and stress release to forecast earthquakes. Here's a great review by Jean-Philippe discussing principles of fault slip budget determination and observational constraints on seismic and aseismic slip \cite{annurev:/content/journals/10.1146/annurev-earth-060614-105302}.


\section{Velocity weakening/strengthening}
Understanding the friction behavior along the fault is the key to understanding the different behaviors of seismic and aseismic slips.
Friction is often associated with the velocity of the fault displacement.
Based on the different responses to sliding velocity, regions on the fault can be classified into two types: velocity weakening and velocity strengthening.
\subsection{Veclocity weakening region}
In a velocity weakening region, the friction resistance between the fault surfaces decreases with an increasing slip velocity.
In other words, as the sliding velocity along the fault increases, the friction strength decreases.

This phenomenon is crucial in earthquake dynamics because it promotes the instability of the fault and facilitates the rapid release of accumulated stress during earthquakes.
When the friction resistance decreases with increasing velocity, the fault becomes more prone to slip suddenly and can generate earthquake waves.
Velocity weakening regions are often associated with materials or conditions that exhibit unstable slip behavior, such as fault gouge, pore pressure, and fluids.

\subsection{Velocity strengthening region}
In contrast to velocity weakening region, a velocity strengthening region is characterized by an increase in frictional resistance with increasing slip velocity.
The velocity strengthening regions tend to promote stable fault behavior, resisting slip and preventing rapid release of stress that leads to earthquakes.
Instead, fault slip in these regions may occur aseismically.

Velocity strengthening is typically observed in the shallow crust and at greater depths where ductile deformation mechanisms dominate. 
Conditions such as high confining pressure and high temperature, which promote stable creep and plastic deformation, contribute to this behavior.
\section{Rate-and-state friction law}
% \subsection{Rate-and-state friction law}
Since the recognition that earthquakes probably represent frictional slip instabilities in the 1960s, interest in determining frictional properties has been increased.
The stability of frictional sliding depends on whether frictional resistance increases or drops during slip.
Laboratory experiments have shown that frictional sliding is mainly a rate-dependent process in a steady-state regime with constant stress and steady velocity.
Due to this, a state variable needs to be introduced to describe
\begin{itemize}
    \item the transient behavior observed in non-steady-state experiments, denoted as $a$. This parameter presents the direct effect of the fault slip rate on the frictional resistance. In many rate-and-state friction laws, an increase in slip rate tends to increase the frictional resistance, and $a$ quantifies the strength of this effect. A higher value of $a$ means that the frictional resistance increases more rapidly with slip rate
    \item healing in hold-and-stick experiments denoted as $b$. This parameter represents the evolutionary effect of the fault slip on the frictional resistance. It quantifies how the frictional resistance evolves over time due to slip history. A positive value of $b$ means that slip tends to decrease the frictional resistance over time, making fault slip easier in the future. A negative value of $b$ implies the opposite, where slip tends to increase the frictional resistance over time, making fault slips harder.
\end{itemize}
These parameters are often determined empirically through laboratory experiments or field study. They play a crucial role in modeling the dynamics of earthquakes.

The formulation of such rate-and-state variable has significantly simplified the impacts of several parameters from rheology on the slip rate, which enables the development of numerical models to simulate earthquake cycles.

For most materials, it has been revealed by laboratory experiments on frictional sliding that the following conditions exist:

\begin{itemize}
    \item The resistance to sliding depends on the sliding rate at steady rate, along with a logarithmic dependency of the coefficient of friction on the slip rate
    \item The resistance to sliding increases to a transient peak value when the imposed slip rate is suddenly changed, with the peak value being a logarithmic function of the slip rate
    \item The friction coefficient is approximately a linear function with the logarithmic of the time in hold-and-slip experiments.
\end{itemize}

Laboratory measurements at slow sliding rates can be reproduced relatively well with a rate-and-state formalism on the order of microns per second.
Various laws have been proposed (\cite{https://doi.org/10.1029/JB084iB05p02161, https://doi.org/10.1029/JB084iB05p02169,https://doi.org/10.1029/JB088iB12p10359,annurev:/content/journals/10.1146/annurev.earth.26.1.643}).

One common such law is called aging law
\begin{align}
    \mu &= \mu_* + (a - b) \ln \frac{V}{V_*} \\
    \frac{d\theta}{dt} &= 1 - \frac{V\theta}{D_c}
\end{align}
At steady state, the law is purely rate dependent
\begin{equation}
    \mu_{ss} = \mu_{*} + (a - b) \ln \frac{V}{V_*}
\end{equation}

For a single-degree-of-freedom system such as a spring-and-slider system, the stability analysis shows that the slip can be stable only if $a - b > 0$ and that unstable slip requires that $a - b < 0$  
%(\cite{Priestley_1990}). 
For unstable slip to occur, it requires $a - b$ that is smaller than a critical negative value, defining an intermediate domain of conditional stability.
For a crack with size L embedded in an elastic medium with shear modulus $G$, the condition for unstable slip is 
\begin{equation}
    a - b < - \lambda \frac{GD_c}{L\sigma_n'}
\end{equation}
where $\lambda$ is on the order of unity. 
$\sigma_n'$ is the effective normal stress with $\sigma_n' - \sigma_n - P$ where $P$ is pore pressure.
In the limit when the pore pressure becomes near lithostatic, the critical value becomes infinite.
This implies that high pore pressure should promote stable slip through the reduction of the effective normal stress.
The rate-and-state friction and many definitions here are important concepts in the earthquake cycle simulations that are going to be discussed in later chapters.

\cite{}