\section{SBP-SAT methods}


SBP methods approximate partial derivatives using one-sided differences at all points close to the boundary node, generating a matrix approximating a partial derivative operator. In this work we focus on second-order derivatives which appear in \eqref{eqn:2Dvar}, however the matrix-free methods we derive are applicable to any second-order PDE. We consider SBP finite-difference approximations to boundary-value problem \eqref{eqn:2Dvar}, i.e. on the square computational domain $\bar\Omega$; solutions on the physical domain $\Omega$ are obtained by the inverse coordinate transformation.

In this work, we focus on SBP operators with second-order accuracy which contains abundant complexity at domain boundaries to enable insight into implementation design extendable to higher-order methods.  To provide background on the SBP methods we first describe the 1D operators, as Kronecker products are used to form their multi-dimensional counterparts. 


\subsubsection{1D Operators}

We discretize the spatial domain $-1\leq r \leq 1$ with $N+1$ evenly spaced grid points $r_i = -1 + ih, i=0, \dots, N$ with grid spacing $h = 2/N$. A function $u$ projected onto the computational grid is denoted by $\boldsymbol{u} = [u_0, u_1, \dots, u_N]^T$ and is often taken to be the interpolant of $u$ at the grid points. We define the grid basis vector $\vec{e}_j$ to be a vector with value 1 at grid point $j$ and 0 for the rest, which allows us to extract the jth component: $u_j = \vec{e}_j^T\vec{u}$.  

\begin{definition}[First Derivative]
  A matrix $\boldsymbol{D}_{r}$ is an SBP approximation to the first derivative operator $\partial /\partial
  r$ if it can be decomposed as $\boldsymbol{H}\boldsymbol{D}_{r} = \boldsymbol{Q}$ with $\boldsymbol{H}$ being SPD and $\boldsymbol{Q}$ satisfying $\vec{u}^{T}(\boldsymbol{Q} +
  \boldsymbol{Q}^{T})\vec{v} = u_{N}v_{N} - u_{0}v_{0}$.
\end{definition}
%
\noindent Here, $\boldsymbol{H}$ is a diagonal quadrature matrix and  $\boldsymbol{D}_{r}$ is the standard central finite difference operator in the interior which transitions to one-sided at boundaries.
The reason why the operator $\emph{D}_x$ is called SBP is that it mimics the integration-by-part property 
\begin{equation}
  \int_0^1 u \frac{\partial v}{\partial x} + \int_0^1 \frac{\partial u}{\partial x} v = uv\bigg|_0^1,
  \label{eqn:1d_ibp}
\end{equation}

\noindent in a discrete form

\begin{equation}
  \vec{u}^T \boldsymbol{H} \boldsymbol{D}_{x}\vec{v} + \vec{u}^T \boldsymbol{D}_{x}^{T} \boldsymbol{H}\vec{v}
  =
  \vec{u}^T\left( \boldsymbol{Q} + \boldsymbol{Q}^T\right)\vec{v}
  =
  u_{N}v_{N} - u_{0}v_{0}.
  \label{eqn:1d_sbp}
\end{equation} 

\begin{definition}[Second Derivative]
  Letting $c = c(r)$ denote a material coefficient, we define matrix $\boldsymbol{D}^{(c)}_{rr}$ to be an SBP approximation to
  $\frac{\partial }{\partial r}\left(c \frac{\partial}{\partial r}\right)$ if it
  can be decomposed as $\boldsymbol{D}^{(c)}_{rr} =\boldsymbol{H}^{-1}(-\boldsymbol{M}^{(c)} +
  c_N\vec{e}_{N}\vec{d}_{N}^{T} -c_0\vec{e}_{0}\vec{d}_{0}^{T})$ where
  $\boldsymbol{M}^{(c)}$ is SPD and $\vec{d}_{0}^{T}\vec{u}$ and
  $\vec{d}_{N}^{T}\vec{u}$ are approximations of the first derivative of $u$ at
  the boundaries.
\end{definition}
%
Similarly, the operator $\boldsymbol{D}_{xx}^{(c)}$ mimics the integration-by-parts property
\begin{equation}
  \int_0^1 u \frac{\partial}{\partial x}\left(c\frac{\partial v}{\partial x}\right)
 + \int_0^1 \frac{\partial u}{\partial x} c\frac{\partial v}{\partial x}
  =
    u c \frac{\partial v}{\partial x} \bigg|_0^1,
    \label{eqn:2d_ibp}
\end{equation}
in a discrete form
\begin{equation}
  \vec{u}^T \boldsymbol{H} \boldsymbol{D}^{(c)}_{xx}\vec{v}
  +\vec{u}^T\boldsymbol{A}^{(c)}\vec{v}
  =
    c_{N}u_{N} \vec{d}_{N}^{T}\vec{v}
  - c_{0}u_{0} \vec{d}_{0}^{T}\vec{v}.
  \label{eqn:2d_sbp}
\end{equation}



With these properties, both $\boldsymbol{D}_{r}$ and $\boldsymbol{D}^{(c)}_{rr}$ mimic integration-by-parts in a discrete form which enables the proof of discrete stability \cite{MN04, Mattsson2009}. 


$\boldsymbol{D}^{(c)}_{rr}$ is a centered difference approximation within the interior of the domain, but includes approximations at boundary points as well. For illustrative purposes alone, if $c = 1$ (e. g. a constant coefficient case), the matrix is given by 
\[
    \boldsymbol{D}^{(c)}_{rr} = \frac{1}{h^2}\begin{bmatrix}
    1 & -2 & 1\\
    {\color{red}1} & {\color{red}-2} & {\color{red}1}\\
    & {\color{red}\ddots} &{\color{red}\ddots} & {\color{red}\ddots}\\
    & & {\color{red}1} &{\color{red}-2} &{\color{red} 1}\\
    & & 1 &-2& \,1 
    \end{bmatrix},
\]
which, as highlighted in red, resembles the traditional (second-order-accurate) Laplacian operator in the domain interior. 




\subsubsection{2D SBP Operators}\label{2d_sbp}
The 2D domain $\bar\Omega$ is discretized using $N+1$ grid points in each direction, resulting in an  $(N+1) \times (N+1)$
grid of points where grid point $(i,j)$ is at $(x_i, y_j) = (-1 + ih, -1 + jh)$ for $0
\leq i, j \leq N$ with $h = 2/N$. Here we have assumed equal grid spacing in each direction, only for notational ease; the generalization to different
numbers of grid points in each dimension does not
impact the construction of the method and is implemented in our code. A 2D grid function $\boldsymbol{u}$ is ordered lexicographically and we let $\boldsymbol{C}_{ij} = \text{ diag}(\boldsymbol{c}_{ij})$ define the diagonal matrix of coefficients, see \cite{Kozdon2020HybridizedSF}.

In this work we imply summation notation whenever indices are repeated. Multi-dimensional SBP operators are obtained by applying the Kronecker product to 1D operators, for example, the 2D second derivative operators are given by
\begin{align}\label{eqn: D2}
  \frac{\partial }{\partial i} c_{ij} \frac{\partial }{\partial j}&\approx \tilde{\boldsymbol{D}}^{c_{ij}}_{ij} \notag \\&= (\boldsymbol{H} \otimes \boldsymbol{H})^{-1}\left[- \tilde{\boldsymbol{M}}^{(c_{ij})}_{ij} + \boldsymbol{T}\right],
\end{align}
for $i, j \in \{r, s\}$.   
Here $\tilde{\boldsymbol{M}}^{(c_{ij})}_{ij}$ is the sum of SPD matrices approximating integrated second derivatives (i.e. sum over repeated indices $i, j$) for example $\int_{\bar\Omega} \frac{\partial}{\partial r} c_{rr} \frac{\partial}{\partial r}  \approx  \tilde{\boldsymbol{M}}^{(c_{rr})}_{rr}$ and matrix $\boldsymbol{T}$ involves the boundary derivative computations, see \cite{Erickson2022} for complete details.
 

\subsubsection{SAT Penalty Terms}
SBP methods are designed to work with various impositions of boundary conditions that lead to provably stable methods, for example through weak enforcement via the simultaneous-approximation-term (SAT) ~\cite{CarpenterGottliebAbarbanel1994} which we adopt here. As opposed to traditional finite difference methods that ``inject" boundary data by overwriting grid points with the given data, the SAT technique imposes boundary conditions weakly (through penalization), so that all grid points approximate both the PDE and the boundary conditions up to a certain level of accuracy.  The combined approach is known as SBP-SAT.
Where traditional methods that use injection or strong enforcement of boundary/interface conditions destroy the discrete integration-by-parts property, using SAT terms enables proof of the method's stability (a necessary property for numerical convergence)~\cite{Mattsson2003}.

\subsection{An example of the SBP-SAT technique for PDE}
We use the following example from \cite{RUGGIU2018216} to showcase an example of applying the SBP-SAT method for PDEs. Let's consider the advection problem in 1D.
\begin{align}
    \begin{split}
    \boldsymbol{u}_t + \boldsymbol{u}_x &= 0, \quad 0 < x < 1,  t > 0 \\
    \boldsymbol{u}(0,t) &=  \boldsymbol{g}(t),  \quad  t > 0 \\
    \boldsymbol{u}(x,t) &= \boldsymbol{h}(x), \quad 0 < x < 1
    \end{split}
     \label{eqn:1d_sbp_problem}
\end{align}

where both g and h are known for initial and boundary conditions. The problem \autoref{eqn:1d_sbp_problem} has an energy-estimate and is well-posed. We can easily learn that the analytical solution for this equation is a right-traveling wave.

We discretize 1D domain with $N+1$ points in a uniform grid on $[0,1]$ using the method described in \autoref{section:1d-sbp}. By applying SBP-SAT discretization in space to \autoref{eqn:1d_sbp_problem}, we get
\begin{align}
    \begin{split}
        \boldsymbol{u}_t + D_1\boldsymbol{u} &= P^{-1}\sigma(\boldsymbol{u}_0 - \boldsymbol{g}) \boldsymbol{e}_0, \quad t > 0\\
        \boldsymbol{u}(0) &= \boldsymbol{h}
    \end{split}
    \label{eqn:1d_sbp_problem_discretized}
\end{align}
where $\boldsymbol{u} = [u_0,\dots,u_N]^T, \boldsymbol{h} = [h_0,\dots,h_N]^T, \sigma \in \mathbb{R}$ is a penalty parameter which is determined through stability condition. $\boldsymbol{e}_0 = [1,0,\dots,0]^T\in\mathbb{R}^{N+1}$. To determine the value for $\sigma$ so that the problem \autoref{eqn:1d_sbp_problem_discretized} is strongly stable, we have 
\begin{align}
    \begin{split}
        ||\boldsymbol{u}(t)||^2 \leq K(t)(||\boldsymbol{h}||^2 + \underset{\tau \in [0,t]}{\text{max}}|\boldsymbol{g}(\tau)|^2)
    \end{split}
    \label{eqn:stability}
\end{align}
The $K(t)$ in \autoref{eqn:stability} is independent of the data and bounded for any finite $t$ and meshsize $\Delta x$. Further details about $K(t)$ are given in \cite{SVARD201417,gustafsson1995time}.
Applying the energy method by multiplying the equation \autoref{eqn:1d_sbp_problem_discretized} with $\boldsymbol{u}^TP$ and adding the transpose with the SBP property \autoref{eqn:1d_sbp}, we find
\begin{equation}
    \frac{d}{dt}||\boldsymbol{u}||_P^2 = - \frac{\sigma^2}{1+2\sigma}\boldsymbol{g}^2 - \boldsymbol{u}_N^2 + \frac{[(1+2\sigma)\boldsymbol{u}_0 - \sigma\boldsymbol{g}]^2}{1+2\sigma}
\end{equation}
By time-integration, this leads to an estimate of the form \autoref{eqn:stability} for $\sigma < -1/2$.

\subsection{Poisson's equation with SBP-SAT Methods}
We consider the 2D Poisson equation on the unit square $\Omega$ with both Dirichlet and Neumann conditions for generality, as each appears in earthquake problems (e.g. Earth's free surface manifests as a Neumann condition, and the slow motion of tectonic plates is usually enforced via a Dirichlet condition). This is an important and necessary first step before additional complexities such as variable material properties, complex geometries, and fully 3D problems. The governing equations are given by
\begin{subequations}\label{eqn:poisson_eqn}
\begin{alignat}{3}
-\Delta u &= f,  &&\quad{\text {for } (x, y) \in \Omega}, \\ 
u &= g_{\text{W}},  &&\quad x = 0,\\ 
u&= g_{\text{E}}, &&\quad x = 1,\\ 
\boldsymbol{n} \cdot \nabla u&=g_{\text{S}}, &&\quad y = 0, \\
\boldsymbol{n} \cdot \nabla u&=g_{\text{N}}, &&\quad y = 1,
\end{alignat}
\end{subequations}
where $\Delta u = \frac{\partial^2 u}{\partial x^2} + \frac{\partial^2 u}{\partial y^2}$, the field $u(x, y)$ is the unknown particle displacement, the scalar function $f(x,y)$ is the source function,
 and vector $\boldsymbol{n}$ is the outward pointing normal to the domain boundary $\partial\Omega$. The $g$'s represent boundary data on the west, east, south, and north boundaries.
 
 
The SBP-SAT discretization of (\ref{eqn:poisson_eqn}) is given by
\begin{equation}\label{eqn:discrete}
-\boldsymbol{D}_2\boldsymbol{u} = \boldsymbol{f} + \boldsymbol{b}^{N} + \boldsymbol{b}^{S} + \boldsymbol{b}^{W} + \boldsymbol{b}^{E},
\end{equation}
%
where $\boldsymbol{D}_2 = \left(\boldsymbol{I} \otimes \boldsymbol{D}_{xx}\right) +  \left(\boldsymbol{D}_{yy}  \otimes \boldsymbol{I}\right)$ is the discrete Laplacian operator and $\boldsymbol{u}$ is the grid function approximating the solution, formed as a stacked vector of vectors. The SAT terms $\boldsymbol{b}^{N}, \boldsymbol{b}^{S}, \boldsymbol{b}^{W}, \boldsymbol{b}^{E}$ enforce all boundary conditions weakly. To illustrate the structure of these vectors, the SAT term enforcing Dirichlet data on the west boundary is given by 
\begin{align}
\boldsymbol{b}^{W} &= \alpha \left(\boldsymbol{H}^{-1} \otimes \boldsymbol{I}\right) (\boldsymbol{E}_\text{W}\boldsymbol{u} - \boldsymbol{e}^T_\text{W}\boldsymbol{g}_\text{W}) \\ 
&- \left(\boldsymbol{H}^{-1}\boldsymbol{e_0}\boldsymbol{d_0}^T \otimes \boldsymbol{I}\right) (\boldsymbol{E}_\text{W}\boldsymbol{u} - \boldsymbol{e}^T_\text{W}\boldsymbol{g}_\text{W}),
\end{align}
where $\alpha$ again represents a penalty parameter, $\boldsymbol{E}_\text{W}$ is a sparse boundary extraction operator, and $\boldsymbol{e}^T_\text{W}$ is an operator that lifts the boundary data to the whole domain. Details of all the SAT terms can be found in \cite{Erickson2014}.
System (\ref{eqn:discrete}) can be rendered SPD by multiplying from the left by $\left(\boldsymbol{H} \otimes \boldsymbol{H}\right)$,  producing the sparse linear system $\boldsymbol{Au} = \boldsymbol{b}$.