HYPRE is a software library of high performance numerical algorithms including preconditioners and solvers for large, sparse linear systems of equations on massively parallel computers \cite{Falgout2006TheDA}. 
The HPYRE library was created to provide users with advanced parallel preconditioners. It features parallel multigrid solvers for both structured and unstructured grid problems.
These solvers are called from application code via HYPRE's conceptual linear system interfaces \cite{10.5555/1125403.1125423}, which allow a variety of natural problem descriptions.

Key features of PETSc include:
\begin{itemize}
    \item Scalable preconditioners: HYPRE contains several families of preconditioners focused on scalable solutions of very large linear systems.
    HYPRE includes "grey box" algorithms including structured multigrid that use more than just the matrix to solve certain classes of problems more efficiently.
    \item Common iterative methods: HYPRE provides several common Krylov-based iterative methods in conjunction with scalable preconditioners. 
    This includes methods for symmetric matrices such as Conjugate Gradient (CG) and nonsymmetric matrices such as GMRES.
    \item Grid-centric interfaces for complicated data structures and advanced solvers: HYPRE has improved usability from earlier generations of sparse linear solver libraries in that users do not have to learn complicated sparse matrix data structures.
    HYPRE builds these data structures for users through a variety of conceptual interfaces for different classes of users.
    These include stencil-based structured/semi-structured interfaces most suitable for finite difference methods, unstructured interfaces for finite element methods, and linear algebra based interfaces for general applications.
    Each conceptual interface provides access to several solvers without the need to manually write code for new interfaces.
    \item User options for beginners through experts: HYPRE allows users with various levels of expertise to write their code easily. The beginner users can set up runnable code with a minimal amount of effort. Expert users can take further control of the solution process through various parameters
    \item Configuration options for different platforms: HYPRE allows a simple and flexible installation on various computing platforms. Users have options to configure for different platforms during the installation.
    Additional options include debug mode which offers more info and optimized mode for better performance.
    It also allows users to change different libraries such as MPI and BLAS.
    \item Interfaces to multiple languages: HYPRE is written in C, but it also provides an interface for Fortran users.
\end{itemize}