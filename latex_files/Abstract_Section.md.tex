Earthquakes often occur on complex faults of multiscale physical features, with different time scales between seismic slips and interseismic periods for multiple events.
Single event, dynamic rupture simulations have been extensively studied to explore earthquake behaviors on complex faults, however, these simulations are limited by artificial prestress conditions and earthquake nucleations.
Over the past decade, significant progress has been made in studying and modeling multiple cycles of earthquakes through collaborations in code comparison and verification.
Numerical simulations for such earthquakes lead to large-scale linear systems that are difficult to solve using traditional methods in this field of study.
These challenges include increased computation and memory demands.
In addition, numerical stability for simulations over multiple earthquake cycles requires new numerical methods.
Developments in High performance computing (HPC) provide tools to tackle some of these challenges.
HPC is nothing new in geophysics since it has been applied in earthquake-related research including seismic imaging and dynamic rupture simulations for decades in both research and industry.
However, there's little work in applying HPC to earthquake cycle modeling. 
This dissertation presents a novel approach to applying the latest advancements in HPC and numerical methods to solving computational challenges in earthquake cycle simulations.